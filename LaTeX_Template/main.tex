\documentclass[a4paper, 12pt]{article}
\def\Author{...}
\def\Matrikelnummer{123456}
\def\Semester{ETMT 1234}
\def\Modul{Werkstofftechnik}
\def\Fachsemester{03}
\def\Aktivitaet{Laborübung}
\def\Thema{Zugversuch}
\def\Versuch{Versuch x}
\def\Betreuer{Dpl. Ing. ...}
\def\Durchfuehrungsdatum{02.03.2022}
\def\Ort{Werkstoff Labor}
% defines the stretch of the table of contents
\def\TOCstretch{2}
% Coding
\usepackage[utf8]{inputenc}
% Explicit mention of the font
\usepackage[T1]{fontenc} 
% \usepackage{helvet}
% \usepackage[onehalfspacing]{setspace}
% Language
\usepackage[ngerman]{babel} 
% always needed for embedding graphics
\usepackage{graphicx} 
% If you want to check the layout, you can insert text here with \blindtext
\usepackage{blindtext} 
% For the distance between 2 paragraphs
\usepackage{parskip}
% Exact setting of parskip
\setlength{\parskip}{12pt plus80pt minus10pt} 
% With \todo{} insert todo
\usepackage{easy-todo} 
% For proper quotation marks
\usepackage{csquotes}
% For a German formatting of the submission date / affidavit
\usepackage[iso, german]{isodate} 
% Beaver backend for bibliography
\usepackage[style=apa, backend=biber, sortlocale=deDE]{biblatex} 
% Incorporating the literature
\addbibresource{literature/bibliography.bib}
% Customize language settings in the bibliography
\DeclareLanguageMapping{german}{german-apa}  
% For fine-tuning punctuation
\usepackage[activate={true,nocompatibility},
	final,
	tracking=true,
	kerning=true,
	expansion=true,
	spacing=true,
	factor=1050,
	stretch=25,
	shrink=10]{microtype}
% other packages
\usepackage{booktabs}
\usepackage{appendix}
\usepackage[rflt]{floatflt}
\usepackage{fancyvrb}
% Clickable but unmarked links in PDF
\usepackage[final]{pdfpages}
\usepackage[hidelinks, bookmarks, bookmarksopen, bookmarksdepth=2]{hyperref}
% \usepackage{bookmark}
\usepackage[nottoc]{tocbibind}
\usepackage{setspace}
% For nicer headers / footers and footnotes.
\usepackage{fancyhdr} 
% Margins
\usepackage[head=40pt, foot=40pt, right=2.5 cm, left=2.5 cm, top=2.5 cm, bottom=3 cm]{geometry} 
\usepackage{pbox}
\usepackage{tabulary}
\usepackage{tabularx}
\usepackage{placeins}
\usepackage{enumitem}
\usepackage{datetime}
\usepackage{siunitx}

\input{settings/page}

\begin{document}
\newgeometry{margin=2.5cm}
\begin{titlepage}
\thispagestyle{empty}
\newcommand{\HRule}{\rule{\linewidth}{0.5mm}}
\hspace{1cm}
\center

\begin{figure}
    \centering
    \includegraphics{images/PHWTLogo.jpg}\\
\end{figure}
\textsc{\Large \Modul}\\[0.8cm]
\MSonehalfspacing
\textsc{\Large \Aktivitaet}\\[1.0cm]

\HRule\\[1.4cm]
\MSdoublespacing
{ \huge \bfseries \Thema}\\[0.2cm]
{ \large \Versuch}\\[0.3cm] 
\HRule \\[1.0cm]
\MSonehalfspacing

\begin{minipage}[t]{0.8\textwidth}
	\begin{itemize}
	\item[\emph{Semester:}] \Semester
	\item[\emph{Ersteller:}] \Author
	\item[\emph{Matr.-Nr.:}] \Matrikelnummer
	\item[\emph{Fachsemester:}] \Fachsemester
	\item[\emph{Modul:}] \Modul
	\item[\emph{Zuständiger Betreuer:}] \Betreuer
	\item[\emph{Durchgeführt am:}] \Durchfuehrungsdatum
	\item[\emph{Ort:}] \Ort
	\end{itemize}
\end{minipage}

\vspace{2.9cm}

\flushright \emph{Abgabedatum:} \today
\end{titlepage}
\restoregeometry

\setstretch{\TOCstretch}
\microtypesetup{protrusion=false}
\pdfbookmark{\contentsname}{toc}
\tableofcontents
\microtypesetup{protrusion=true}
\thispagestyle{empty}

\MSonehalfspacing
\newpage
\setcounter{page}{1}
\pagestyle{fancy}
\setcounter{page}{1}
\newcommand{\groesse}{13cm}

\section{Einleitung}
Beispiel um ein Bild einzufügen mit einer spezifischen Größe

\begin{figure}[!ht]
    \centering
    \includegraphics[width=\groesse]{images/PHWTLogo.jpg}
    \caption{Beschreibung}
    \label{fig:referenz}
\end{figure}
\FloatBarrier

\subsection{.}
Hier kann etwas stehen ... \par
% bibliography reference 
\nocite{Abadi.2016} % not visible in document
\cite{Abadi.2016}  % visible in document

\subsubsection{-}
Hier kann etwas stehen ... \par
\nameref{marker}

\newpage
\section{Hauptteil}
Beispiel einer Auflistung

\begin{itemize}
    \item Hallo
    \item Hello
\end{itemize}

\begin{enumerate}[noitemsep]
\centering
    \item Hallo
    \item Hello
\end{enumerate}

\begin{description}
  \item[Erster Punkt]~\par
  \begin{itemize}
     \item hier kann etwas stehen
     \item ...
     \item sample text
  \end{itemize}
  \item[Zweiter Punkt]~\par
  \begin{itemize}
      \item hello
  \end{itemize}
  \item[Dritter Punkt]~\par
\end{description}

\subsection{..}
Beispiel für eine Tabelle

\bgroup
\def\arraystretch{1.5}
\begin{table}[!ht]
    \large
    \centering
    \begin{tabular}{|c|c|c|}
    \hline
    AlMgSi1,2 & C15 & S235 \\
    \hline 
    AlMg5Mm & C45 + 5 & S355 \\
    \hline
    AlMg0,8Si1,5 & 42CrMo4 & H340LAD \\
    \hline
    Al99 & X5CrNi18-10 & S500 \\
    \hline
    AlCu0,8Mg & X10CrNi18-10 & H380LAD \\
    \hline
    AlCu4PbMgMn & X8CrNiMo20-15 & S450J2\\
    \hline
    \end{tabular}
    \caption{Zu untersuchende Werkstoffe}
    \label{tab:tabelle1}
\end{table}
\egroup

\newpage
\subsubsection{--}
Weiteres Beispiel für eine Tabelle

\bgroup
\def\arraystretch{1.5}
\begin{table}[!ht]
    \small
    \centering
    \begin{tabular}{l|c|r}
    AlMgSi1,2 & C15 & S235 \\
    \hline 
    AlMg5Mm & C45 + 5 & S355 \\
    \hline
    AlMg0,8Si1,5 & 42CrMo4 & H340LAD \\
    \hline
    Al99 & X5CrNi18-10 & S500 \\
    \hline
    AlCu0,8Mg & X10CrNi18-10 & H380LAD \\
    \hline
    AlCu4PbMgMn & X8CrNiMo20-15 & S450J2\\
    \end{tabular}
    \caption{Zu untersuchende Werkstoffe}
    \label{tab:tabelle2}
\end{table}
\egroup
\FloatBarrier

\subsubsection{~~}
Tabelle mit fester Spalten Breite

\bgroup
\def\arraystretch{1.5}
\begin{table}[!ht]
    \centering
    \begin{tabularx}{\textwidth}{X|X}
    Vorteil & Nachteil \\
    \hline
    Läuft unabhängig vom Programm & Benötigt einen Timer \\
    Genaueres PWM-Signal & Signalausgabe nur auf bestimmten, fest vorgegebenen Pins möglich\\
    Bieter mehr Möglichkeiten bei gleichem Softwareaufwand & Nur Teiler durch 2 möglich \\
    \end{tabularx}
    \label{tab:my_label}
\end{table}
\egroup

\newpage
\section{Schluss}
Beispiel für eine Mathegleichung

\begin{displaymath}
\scalebox{3}{$\epsilon = \frac{\Delta\sigma}{\sqrt{\pi}}$}
\end{displaymath}

\subsection{...}
hoch$_{tief}$ tief$^{hoch}$ 
\[ \int\limits_0^1 x^2 + y^2 \ dx \]

\subsubsection{---}
\[ \sum_{i=1}^{\infty} \frac{1}{n^s} = \prod_p \frac{1}{1 - p^{-s}} \]


\newpage
\printbibliography

\newpage
\addtocontents{toc}{\protect\setcounter{tocdepth}{0}}
\renewcommand{\appendixtocname}{Anhang}
\addappheadtotoc
\renewcommand{\appendixpagename}{Anhang}
\appendices
\appendixpage
\appendixtitleoff
\newcounter{pdfpage}
%%%%%%%%%%%%%%%%%%%%%%%%%%%%%
% Example Appendix with pdf, Page 8
Erster Eintrag: \hyperref[file-1]{Erste Seite}

\includepdf[scale=0.9, frame, pages=3, pagecommand={\thispagestyle{plain}\refstepcounter{pdfpage}\label{file-\thepdfpage}}]{attachments/phwt_style.pdf}

%%%%%%%%%%%%%%%%%%%%%%%%%%%%
\MSonehalfspacing
\newpage
\restoreapp

\newpage
\section*{DIN Normen}

\section*{DIN EN 10002}\label{marker}
Die Versuchsdurchführung erfolgt gemäß DIN EN 10002. Zunächst ist die Probe auszumessen, mit den Vorschriften der Norm zu vergleichen und die Messlänge zu bestimmen.
Danach ist die Probe in die Zugprüfmaschine einzubauen. Die Krafteinleitung erfolgt ausschließlich in axialer Richtung. Zur Messung der Verlängerung wird ein elektronisches
Dehnungsmessgerät, wie in der nachfolgenden Abbildung dargestellt, auf die Probe aufgesetzt mit dem währen der Prüfung kontinuierlich die Probenverlängerung gemessen
wird.


\newpage
\section*{Eigenständigkeitserklärung}
Hiermit versichere ich, dass ich die Hausarbeit selbstständig verfasst und keine anderen als die angegebenen Quellen und Hilfsmittel benutzt habe, alle Ausführungen, die anderen Schriften wörtlich oder sinngemäß entnommen wurden, kenntlich gemacht sind und die Arbeit in gleicher oder ähnlicher Fassung noch nicht Bestandteil einer Studien- oder Prüfungsleistung war.

\vspace{100mm}
% Ort, Datum
\noindent{}STADT, den \today
% Line with 8 cm width
\begin{minipage}[t]{8cm}
\centering \hspace{20mm} \hrulefill \\
% Text under the line
\hspace{20mm}VORNAME NACHNAME
\end{minipage}
\end{document}
